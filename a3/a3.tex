\documentclass[a4paper, 12 pt, conference, onecolumn]{IEEEconf}

\usepackage{siunitx}
\usepackage{graphicx}
\usepackage[utf8]{inputenc}
\usepackage[T1]{fontenc}
\usepackage{subfigure, siunitx, url}
\usepackage{amsmath, amssymb, nicefrac}
\usepackage{pifont}
%\usepackage{tikz}
%\usetikzlibrary{shapes,arrows,fit,trees,positioning,chains,calc,intersections}

%%% abbreviations to use (guarantee correct spaces between characters)
\newcommand{\ie}{i.e.\ }
\newcommand{\Ie}{I.e.\ }
\newcommand{\eg}{e.g.\ }
\newcommand{\Eg}{E.g.\ }
\newcommand{\cf}{cf.}

%%% commands for referencing (guarantee homogeneous writing style)
\newcommand{\referenceChapter}[1]{Chapter \ref{#1}}
\newcommand{\referenceSection}[1]{Section \ref{#1}}
\newcommand{\referenceFigure}[1]{Figure \ref{#1}}
\newcommand{\referenceEquation}[1]{Eq.~(\ref{#1})}
\newcommand{\referenceTable}[1]{Table \ref{#1}}
\newcommand{\referenceListing}[1]{code listing\ref{#1}}
\newcommand{\referenceAlgorithm}[1]{Algorithm \ref{#1}}

\title{\LARGE \bf
MAS Probabilistic Reasoning\\* 
}


\author{Alexander Moriarty\\~\\~
        Probabilistic Reasoning: Assignment 3\\
        Bonn Rhein-Sieg Uni. of Applied Science\\ 
        Sankt Augustin, NRW, Germany\\
        \tt\small alexander@dal.ca
}

\begin{document}

\maketitle

\section{Question 1}
\textbf{13.10 Show the statement ~\ref{eq:q1-1} is equivalent to either ~\ref{eq:q1-2} or ~\ref{eq:q1-3}.}
\begin{eqnarray}
P(A, B \mid C) &=& P(A \mid C) P(B \mid C) \label{eq:q1-1} \\
P(A \mid B, C) &=& P(A \mid C) \label{eq:q1-2} \\
P(B \mid A, C) &=& P(B \mid C) \label{eq:q1-3}
\end{eqnarray}
\noindent\makebox[\linewidth]{\rule{\linewidth}{0.1pt}} 
\setcounter{equation}{0}
\begin{eqnarray}
P(A, B \mid C) &=& P(A \mid C) P(B \mid C) \\
\frac{P(A, B, C)}{P(C)} &=& P(A \mid C)\frac{P(B, C)}{P(C)} \\
\frac{P(A \mid B, C) P(B, C)}{P(C)} &=& P(A \mid C)\frac{P(B, C)}{P(C)} \\
P(A \mid B, C)\frac{P(B, C)}{P(C)} &=& P(A \mid C)\frac{P(B, C)}{P(C)} \\
P(A \mid B, C) &=& P(A \mid C)
\end{eqnarray}


\setcounter{paragraph}{0}
\setcounter{equation}{0}
\section{Question 2}
\textbf{13.20 For the Wumpus world in Chap. 13.7 compute the term $\sum\limits_{other}P(other)$ for the various pit configurations shown on p. 485}
\begin{eqnarray}
... todo ...
\end{eqnarray}


\end{document}
